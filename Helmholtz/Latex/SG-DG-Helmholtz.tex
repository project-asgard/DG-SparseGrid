%\documentclass[final,leqno]{siamltex704}
%\documentclass[leqno]{siamltex704}

\documentclass[preprint,11pt]{elsarticle}


\usepackage{epsfig,graphicx}
\usepackage{amssymb,amsmath,bm}
\usepackage{float}
\usepackage{tikz}

\usepackage{todonotes}

%\usepackage[notcite,notref]{showkeys}

\newproof{proof}{Proof}
\newtheorem{lemma}{Lemma}[section]
\newtheorem{theorem}{Theorem}[section]
\newtheorem{remark}{Remark}[section]


\newcommand{\bq}{{\bf q}}
\newcommand{\bn}{{\bf n}}
\newcommand{\bx}{{\bf x}}
\newcommand{\bv}{{\bf v}}
\def\bbb{{\bf b}}
\def\T{{\mathcal T}}
\def\E{{\mathcal E}}
\def\V{{\mathcal V}}
\def\W{{\mathcal W}}
\def\l{{\langle}}
\def\r{{\rangle}}
\def\jump#1{{[\![#1[\!]}}
\def\bL{{\bf L}}
\def\bbF{{\bf F}}
\def\bbf{{\bf f}}
\def\bn{{\bf n}}
\def\bq{{\bf q}}
\def\bV{{\bf V}}
\def\bu{{\bf u}}
\def\bv{{\bf v}}
\def\bw{{\bf w}}
\def\br{{\bf r}}
\def\bs{{\bf s}}
\def\bbQ{\mathbb{Q}}
\def\bfQ{\bf{Q}}
\def\bcurl{ \textbf{curl }}
\def\bE{{\bf E}}
\def\bB{{\bf B}}
\def\bx{{\bf x}}
\def\bU{{\bf U}}
\def\bV{{\bf V}}
\def\bJ{{\bf J}}
\def\be{{\bf e}}

\def\ljump{{[\![}}
\def\rjump{{]\!]}}

\def\lavg{{\{\!\{}}
\def\ravg{{\}\!\}}}

\def\aa{\mathfrak{a}}
\def\bbQ{\mathbb{Q}}


\def\3bar{{|\hspace{-.02in}|\hspace{-.02in}|}}

%\setlength{\textwidth}{6truein} \setlength{\textheight}{8truein}
%\voffset=-0.55truein
%\hoffset=-0.65truein

\setlength{\textwidth}{6.5truein} \setlength{\textheight}{8truein}
\voffset=-0.6truein
\hoffset=-0.7truein


\journal{Journal of Computational Physics}

\begin{document}

\begin{frontmatter}

\title{Discontinuous Galerkin Sparse Grids Methods for Helmholtz Equations with Large Wave Numbers}
\tnotetext[t1]{This material is based upon work supported in part by the Laboratory Directed Research and Development program at the Oak Ridge National Laboratory, which is operated by UT-Battelle, LLC., for the U.S. Department of Energy under Contract DE-AC05-00OR22725.} 
%\tnotetext[t2]{The second title footnote which is a longer
%text matter to fill through the whole text width and overflow into another line in the footnotes area of the first page.}

\author[1]{E. D'Azevedo}
\address[1]{Computational and Applied Mathematics Division, Oak Ridge National Laboratory, Oak Ridge, TN 37831, USA}
\ead{dazevedoef@ornl.gov}

\author[2]{David Green}
\address[2]{Fusion \& Materials for Nuclear Systems Division
Oak Ridge National Laboratory, Oak Ridge, TN 37831, USA}
\ead{greendl1@ornl.gov}

\author[3]{Lin Mu\corref{cor1}}%\fnref{fn1}}
\address[3]{Computational and Applied Mathematics Division, Oak Ridge National Laboratory, Oak Ridge,  TN 37831, USA}
\ead{mul1@ornl.gov}

\cortext[cor1]{Corresponding author}


\begin{abstract}
I
\end{abstract}

\begin{keyword}
Helmholtz equations, discontinuous Galerkin method, sparse grids methods.
\end{keyword}

\end{frontmatter}




%===============
% Introduction
%===============
\section{Introduction}\label{Sect:Intro}
Helmholtz equation arise in a number of physical applications, in particular in problems of wave scattering and fluid-solid-interaction.
%----------------------------
% Helmholtz Equations
%----------------------------
In this paper, we shall consider the following Helmholtz equations 
\begin{eqnarray}
-\Delta u - \kappa^2 u &=& f,\text{ in }\Omega \\
\frac{\partial u}{\partial\bn} + i\kappa u &=& g, \text{ on }\Gamma_R\\
u &=&  0,\text{ on }\Gamma_D,
\end{eqnarray}
where $\kappa$ is the wave number. The quality of discrete numerical solutions to the Helmholtz equation depends significantly on the physical parameter $\kappa$. It is clear and well known that the stepwidth $h$ of meshes for finite element or finite difference computations should be adjusted to the wavenumber $\kappa$. In practice, one usually follows a "rule of the thumb" of the following form
$$\kappa h = \text{const}.$$
In computations with low wavenumber, this rule leads to sufficiently correct results. The quality of numerical results, however, deteriorates if the wavenumber $\kappa$ increases. On the other hand, the erros are bounded on a series of meshes with $\kappa^3h^2\approx$const. It is shown that, if $\kappa^2h$ is sufficiently small, the error in $H^1$-seminorm satisfies a quasioptimal estimate
$$|u-u_{h}|_1\le C\inf_{v\in V_h}|u-v|_1,$$
where $V_h$ is the finite element subspace and $C$ is a constant that does not depend on $\kappa$ and $h.$ However, the assumption on $\kappa^2h$ is unsatisfactory from a practical point of view since it generally holds on very fine mesh only. It is shown in previous reference that the relative error of the FE-solution in $H^1$-seminorm generally can be written as
$$|u-u_h|_1\le C_1kh+C_2k^3h^2.$$
The first term on the right hand side reflects the approximation error which is of local character; the second part is due to numerical pollution, which is a global effect that can be connected to a phase lead of the numerical solution. 

\section{Numerical Experiment}\label{Sect:NumTest}
\subsection{Test 1}
In this test, the analytical solution is chosen as $u=e^{i\kappa x}$ and the domain is $[0,1]$. We shall compare two different numerical schemes:
\begin{itemize}
\item New Formulation
\begin{eqnarray}
\mathcal{A}(u,v) = B(u,v) - C(u,v)+ \text{i}\frac{\sigma}{h}D(u,v) + \text{i}\kappa E(u,v) + \text{i}J(u,v),\label{scheme1}
\end{eqnarray}
where 
\begin{eqnarray*}
B(u,v)&=&\sum_{K\in\mathcal{T}_h}(\nabla u,\nabla v) - \kappa^2(u,v),\\
C(u,v)&=&\sum_{e\in\mathcal{E}_h^0}\int_e \left(\lavg\nabla u\ravg\ljump v\rjump + \lavg\nabla v\ravg\ljump u\rjump\right) ds,\\
D(u,v) &=& \sum_{e\in\mathcal{E}_h^0}\int_e \ljump u\rjump\cdot\ljump v\rjump ds,\\
E(u,v) &=& \sum_{e\in\Gamma_R}\int_e uvds\\
J(u,v) &=& \sum_{e\in\mathcal{E}_h^0}\sum_{0< q\le p}\int_e\ljump u^q\rjump\cdot\ljump v^q\rjump ds.
\end{eqnarray*}

\item IPDG scheme
\begin{eqnarray*}
A(u,v) = B(u,v) - C(u,v)+\frac{\sigma}{h}D(u,v) + \text{i}\kappa E(u,v).\label{scheme2}
\end{eqnarray*}


\end{itemize}

The first test is to compare the eigenvalues of the matrices from the bilinear form $\mathcal{A}(u,v)$ and  $A(u,v)$. In this test, we choose $\kappa = 10$ and the eigenvalues are plotted in Figure \ref{Fig:Ex1-Eigs}.
\begin{figure}[H]
\centering
\begin{tabular}{cc}
\includegraphics[width=.49\textwidth,height=.49\textwidth]{./Fig/Test_1D_k10_Method1}
&\includegraphics[width=.49\textwidth,height=.49\textwidth]{./Fig/Test_1D_k10_Method2}\\
(a) & (b)
\end{tabular}
\caption{Example \ref{Test:Ex1}: Illustration of eigenvalues from linear system $\mathcal{A}(u_h,v)$ and ${A}(u_h,v)$.}\label{Fig:Ex1-Eigs}
\end{figure}

\begin{figure}[H]
\centering
\begin{tabular}{cc}
\includegraphics[width=.49\textwidth,height=.45\textwidth]{./Fig/Test_1D_k100_ConvRate}
&\includegraphics[width=.49\textwidth,height=.45\textwidth]{./Fig/Test_1D_k100_CondNumb}\\
(a) & (b)
\end{tabular}
\caption{Example \ref{Test:Ex1}: Illustration of eigenvalues from linear system $\mathcal{A}(u_h,v)$ and ${A}(u_h,v)$.}\label{Fig:Ex1-ConRate}
\end{figure}

\begin{figure}[H]
\centering
\includegraphics[width=.7\textwidth]{./Fig/Test_1D_k1000_CondNumb}
\caption{Example \ref{Test:Ex1}: Illustration of eigenvalues from linear system $\mathcal{A}(u_h,v)$ and ${A}(u_h,v)$.}\label{Fig:Ex1-1000}
\end{figure}





\begin{figure}[H]
\centering
\includegraphics[width=.7\textwidth]{./Fig/Test_1D_Method1_k3h2}
\caption{Example \ref{Test:Ex1}: Illustration of eigenvalues from linear system $\mathcal{A}(u_h,v)$ and ${A}(u_h,v)$.}\label{Fig:Ex1-k3h2}
\end{figure}

%-----------------
% Test 2
%-----------------
\subsection{Test 2}\label{Test:Ex2}
In this test, let $\Omega = (0,1)$ and the exact solution as 
$$u = \frac{1-\cos(\kappa x)-\sin(\kappa)\sin(\kappa x)+\text{i}(1-\cos(\kappa))\sin(\kappa x)}{\kappa^2}.$$
The boundary conditions are given as
\begin{eqnarray*}
u = 0, \text{ if }x=0; \quad \frac{\partial u}{\partial\bn}+\text{i}\kappa u = 0, \text{ if }x=1.
\end{eqnarray*}

\begin{figure}[H]
\centering
\includegraphics[width=.48\textwidth]{./Fig/Eigs-k1000-Deg6-Method1}
\includegraphics[width=.49\textwidth]{./Fig/Eigs-k1000-Deg6-Method2}
%\includegraphics[width=.23\textwidth]{./Fig/Eigs-k1000-Deg10-Method1}
%\includegraphics[width=.23\textwidth]{./Fig/Eigs-k1000-Deg10-Method2}
\caption{Example \ref{Test:Ex2}: Eigenvalues corresponding to Scheme (\ref{scheme1}) and Scheme (\ref{scheme2}) on mesh with Lev = 8, Deg = 5. The $L^2$-errors are 2.86E-2 and 3.93E-2.}%\label{Fig:Ex2-1}
\end{figure}



\begin{figure}[H]
\centering
\includegraphics[width=.45\textwidth]{./Fig/sol-Test2-k32}
\includegraphics[width=.49\textwidth]{./Fig/ConRate-Deg2}
%\includegraphics[width=.32\textwidth]{./Fig/ConRate-Deg4}
\caption{Example \ref{Test:Ex2}: Error profiles and convergence tests for scheme (\ref{scheme1}) with $p = 2$, $p = 3$, $p=4$. The wave numbers are chosen as $\kappa = 2^4,2^5,\cdots,2^9$.}\label{Fig:Ex2-1}
\end{figure}



\begin{figure}[H]
\centering
%\includegraphics[width=.32\textwidth]{./Fig/ConRate-Deg2}
\includegraphics[width=.49\textwidth]{./Fig/ConRate-Deg3}
\includegraphics[width=.49\textwidth]{./Fig/ConRate-Deg4}
\caption{Example \ref{Test:Ex2}: Error profiles and convergence tests for scheme (\ref{scheme1}) with $p = 2$, $p = 3$, $p=4$. The wave numbers are chosen as $\kappa = 2^4,2^5,\cdots,2^9$.}\label{Fig:Ex2-2}
\end{figure}

\begin{figure}[H]
\centering
%\includegraphics[width=.32\textwidth]{./Fig/ConRate-Deg2}
\includegraphics[width=.49\textwidth]{./Fig/ConRate-Deg6}
\includegraphics[width=.49\textwidth]{./Fig/ConRate-Deg7}
\caption{Example \ref{Test:Ex2}: Error profiles and convergence tests for scheme (\ref{scheme1}) with $p = 2$, $p = 3$, $p=4$. The wave numbers are chosen as $\kappa = 2^4,2^5,\cdots,2^9$.}\label{Fig:Ex2-3}
\end{figure}


\begin{figure}[H]
\centering
%\includegraphics[width=.32\textwidth]{./Fig/ConRate-Deg2}
\includegraphics[width=.49\textwidth]{./Fig/ConRate-Deg4}
\includegraphics[width=.49\textwidth]{./Fig/ConRate-Deg5-Method2}
\caption{Example \ref{Test:Ex2}: Error profiles and convergence tests for scheme (\ref{scheme1}) with $p = 2$, $p = 3$, $p=4$. The wave numbers are chosen as $\kappa = 2^4,2^5,\cdots,2^9$.}\label{Fig:Ex2-}
\end{figure}


\begin{figure}[H]
\centering
\begin{tabular}{cc}
%\includegraphics[width=.32\textwidth]{./Fig/ConRate-Deg2}
\includegraphics[width=.49\textwidth]{./Fig/ConvRate-k100-Method1}
&
\includegraphics[width=.49\textwidth]{./Fig/ConvRate-k1000-Method1}\\
(a) & (b)
\end{tabular}
\caption{Example \ref{Test:Ex2}: Error profiles and convergence tests for scheme (\ref{scheme1}) with: (a) $\kappa = 100$, (b) $\kappa = 1000$.}\label{Fig:Ex2-3}
\end{figure}

\subsection{Shifted Laplacian}
In this test, we shall choose the preconditioner as
\begin{eqnarray*}
-\Delta u - \kappa^2 u -\text{i}\epsilon u = f.
\end{eqnarray*}

\begin{figure}[H]
\centering
\begin{tabular}{cc}
%\includegraphics[width=.32\textwidth]{./Fig/ConRate-Deg2}
\includegraphics[width=.49\textwidth]{./Fig/Eig-Method2-kappa}
&
\includegraphics[width=.49\textwidth]{./Fig/Eig-Method2-kappa2}\\
(a) & (b)
\end{tabular}
\caption{(a) $\epsilon = \kappa$, (b) $\epsilon = \kappa^2$.}\label{Fig:Ex3-1}
\end{figure}

\begin{figure}[H]
\centering
\begin{tabular}{c}
%\includegraphics[width=.32\textwidth]{./Fig/ConRate-Deg2}
\includegraphics[width=.6\textwidth]{./Fig/Method1-Lev10-Deg4-BICG}
%&
%\includegraphics[width=.49\textwidth]{./Fig/Eig-Method2-kappa2}\\
%(a) & (b)
\end{tabular}
\caption{(a) $\epsilon = \kappa$.}\label{Fig:Ex3-2}
\end{figure}



%============
% Reference
%============
%\begin{thebibliography}{99}
\begin{thebibliography}{00}

\bibitem{MelenkParsaniaSauter2013}
J.M. Melenk, A. Parsania, S. Sauter. General DG-methods for highly indefinite Helmholtz problems. Journal of Scientific Computing. 57(2013):536-81.

\bibitem{IhlenburgBabuska1997}
F. Ihlenburg, I. Babuska.
Finite Element Solution of the Helmholtz Equation with High Wave Number Part II: The h-p Version of the FEM. SIAM J. Numer. Anal., 34(1997), 315-358.


\end{thebibliography}



\end{document}


